\documentclass[11pt]{article}

% Packages
\usepackage[a4paper, margin=1.5cm]{geometry}
\usepackage{amsmath}
\usepackage{graphicx}
\usepackage{tabularx}
\usepackage{enumitem}
\usepackage{subcaption}

% Define new column types
\newcolumntype{L}{>{\raggedright\arraybackslash}X}
\newcolumntype{C}{>{\centering\arraybackslash}X}

% Define table padding
\renewcommand{\arraystretch}{2}     % increases row height (vertical padding)
\setlength{\tabcolsep}{12pt}        % increases horizontal padding between columns

% Custom Date
\newcommand{\fancydate}{May 19\textsuperscript{th}, 2025}

% Custom title
\title{\Huge\bfseries{Lecture 00}}
\author{\Large{Bryce Benham}}
\date{\fancydate}

\begin{document}

% Title Page
\begin{titlepage}
	\centering
	\vspace*{3cm}

	\rule{\textwidth}{1pt}\par
	\vspace{0.5em}

	{\Huge\bfseries{Lecture 00}\par}
	\vspace{1em}
	{\LARGE{AAA000 - Course Name}\par}

	\vspace{0.5em}
	\rule{\textwidth}{1pt}\par

	\vspace{0.5cm}

	{\Large{Bryce Benham}\par}
	\vspace{0.5em}
	{\large \fancydate \par}

	\vfill
\end{titlepage}

\newpage

\section*{Lecture Syllabus}

\noindent
\makebox[\textwidth][c]{%
	\begin{tabular}{|p{6.65cm}|p{12.35cm}|}
		\hline
		\textbf{Cue} & \textbf{Notes} \\
		\hline

		% Cue column
		Written after class (sometimes during). Remember to only use
		\textbf{short phrases} or \textbf{single words} - not full sentences.

		\begin{itemize}
			\item Key questions
			\item Main ideas/concepts
			\item Important names or dates
			\item Vocabulary terms
			\item Formulas or equations
			\item Triggers for review
			\item Exam/Quiz hints
			\item Connection or analogies
			\item Examples to remember
		\end{itemize}

		After class, \textbf{review your notes and add questions or cues}. Use
		cues to \textbf{quiz yourself} when studying later.
		             &
		% Notes column
		Taken during class, keep bullet points \textbf{clear and concise} but
		sufficiently detailed. Use \textbf{indentation} or \textbf{sub-bullets}
		for related ideas. Highlight or underline \textbf{key terms}. Use
		different colors or fonts if possible.

		\begin{itemize}[left=0pt]
			\item Detailed explanations
			\item Definitions
			\item Examples
			\item Diagrams and charts
			\item Step-by-step processes
			\item Important dates, names and places
			\item Quotes or exact phrases
			\item Comparisons and contrast
			\item Class discussion and questions
			\item Your own paraphrasing
			\item Abbreviations and symbols
			\item Spaces for adding information later
		\end{itemize}
		\\
		\hline
	\end{tabular}%
}

\newpage

% Summary
\noindent
\makebox[\textwidth][c]{%
	\begin{tabular}{|p{16cm}|}
		\hline
		\textbf{Summary} \\
		\hline

		% Start Summary Here!
		Keep it \textbf{short and clear}, ideally 2 - 5 sentences. Use your own
		words to \textbf{solidify understanding}. Write the summary \textbf{after
			class} or \textbf{studying}, not during. Use the summary as a \textbf{quick
			refresher} before exams.

		\begin{itemize}
			\item Key takeaways
			\item Big picture overview
			\item Main answers
			\item Connections
			\item Purpose of the lecture
			\item Personal reflections
			\item Action items
		\end{itemize}

		This page summarizes the key points discussed during the lecture.
		Write a concise overview of the main concepts, including takeaways,
		definitions, or conclusions drawn from the session. This area is also
		useful for self-reflection or noting study reminders.
		\\
		\hline
	\end{tabular}%
}
\end{document}
